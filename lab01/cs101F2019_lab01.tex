\documentclass[11pt]{article}

% NOTE: The "Edit" sections are changed for each assignment

% Edit these commands for each assignment

\newcommand{\assignmentduedate}{September 19}
\newcommand{\assignmentassignedate}{September 12}
\newcommand{\assignmentnumber}{One}

\newcommand{\labyear}{2019}
\newcommand{\labday}{Thursday}
\newcommand{\labtime}{2:30 pm}

\newcommand{\assigneddate}{Assigned: \labday, \assignmentassignedate, \labyear{} at \labtime{}}
\newcommand{\duedate}{Due: \labday, \assignmentduedate, \labyear{} at \labtime{}}

% Edit these commands to give the name to the main program

\newcommand{\mainprogram}{\lstinline{CreditCard}}
\newcommand{\mainprogramsource}{\lstinline{src/main/java/labone/CreditCard.java}}

% Edit these commands to give the name to the test suite

\newcommand{\testprogram}{\lstinline{TestCreditCard}}
\newcommand{\testprogramsource}{\lstinline{src/test/java/labone/TestCreditCard.java}}

% Edit this commands to describe key deliverables

\newcommand{\reflection}{\lstinline{writing/reflection.md}}

% Commands to describe key development tasks

% --> Running gatorgrader
\newcommand{\gatorgraderstart}{\command{gradle grade}}
\newcommand{\gatorgradercheck}{\command{gradle grade}}

% --> Compiling and running and testing program with gradle
\newcommand{\gradlebuild}{\command{gradle build}}
\newcommand{\gradletest}{\command{gradle test}}
\newcommand{\gradlerun}{\command{gradle run}}

% Commands to describe key git tasks

% NOTE: Could be improved, problems due to nesting

\newcommand{\gitcommitfile}[1]{\command{git commit #1}}
\newcommand{\gitaddfile}[1]{\command{git add #1}}

\newcommand{\gitadd}{\command{git add}}
\newcommand{\gitcommit}{\command{git commit}}
\newcommand{\gitpush}{\command{git push}}
\newcommand{\gitpull}{\command{git pull}}

\newcommand{\gitcommitmainprogram}{\command{git commit src/main/java/labone/CreditCard.java -m "Your
descriptive commit message"}}

% Use this when displaying a new command

\newcommand{\command}[1]{``\lstinline{#1}''}
\newcommand{\program}[1]{\lstinline{#1}}
\newcommand{\url}[1]{\lstinline{#1}}
\newcommand{\channel}[1]{\lstinline{#1}}
\newcommand{\option}[1]{``{#1}''}
\newcommand{\step}[1]{``{#1}''}

\usepackage{pifont}
\newcommand{\checkmark}{\ding{51}}
\newcommand{\naughtmark}{\ding{55}}

\usepackage{listings}
\lstset{
  basicstyle=\small\ttfamily,
  columns=flexible,
  breaklines=true
}

\usepackage{fancyhdr}

\usepackage[margin=1in]{geometry}
\usepackage{fancyhdr}

\pagestyle{fancy}

\fancyhf{}
\rhead{Computer Science 101}
\lhead{Laboratory Assignment \assignmentnumber{}}
\rfoot{Page \thepage}
\lfoot{\duedate}

\usepackage{titlesec}
\titlespacing\section{0pt}{6pt plus 4pt minus 2pt}{4pt plus 2pt minus 2pt}

\newcommand{\labtitle}[1]
{
  \begin{center}
    \begin{center}
      \bf
      CMPSC 101\\Data Abstraction\\
      Fall 2019\\
      \medskip
    \end{center}
    \bf
    #1
  \end{center}
}

\begin{document}

\thispagestyle{empty}

\labtitle{Laboratory \assignmentnumber{} \\ \assigneddate{} \\ \duedate{}}

\section*{Objectives}

To learn how to use GitHub to access the files for a course assignment.
Additionally, to learn how to use your laptop's operating system and software
development tools such as a ``terminal window'' and the ``Docker Desktop''. You
will also continue to practice using Slack to support communication with the
technical leaders and the course instructor. Next, you will learn how to
implement a Java program and to write a Markdown document, also discovering how
to use the course's automated grading tool to assess your progress towards
correctly completing the project.

\section*{Suggestions for Success}

\begin{itemize}
  \setlength{\itemsep}{0pt}

\item {\bf Follow each step carefully}. Slowly read each sentence in the
  assignment sheet, making sure that you precisely follow each instruction. Take
  notes about each step that you attempt, recording your questions and ideas and
  the challenges that you faced. If you are stuck, then please tell a teaching
  assistant or instructor what assignment step you recently completed.

\item {\bf Regularly ask and answer questions}. Please log into Slack at the
  start of a laboratory or practical session and then join the appropriate
  channel. If you have a question about one of the steps in an assignment, then
  you can post it to the designated channel. Or, you can ask a student sitting
  next to you or talk with a teaching assistant or the course instructor.

\item {\bf Store your files in GitHub}. Starting with this laboratory
  assignment, you will be responsible for storing all of your files (e.g., Java
  source code and Markdown-based writing) in a Git repository using GitHub
  Classroom. Please verify that you have saved your source code in your Git
  repository by using \command{git status} to ensure that everything is updated.
  You can see if your assignment submission meets the established correctness
  requirements by using the provided checking tools on your local computer and
  in checking the commits in GitHub.

\item {\bf Keep all of your files}. Don't delete your programs, output files,
  and written reports after you submit them through GitHub; you will need them
  again when you study for the quizzes and examinations and work on the other
  laboratory, practical, and final project assignments.

\item {\bf Explore teamwork and technologies}. While certain aspects of the
  laboratory assignments will be challenging for you, each part is designed to
  give you the opportunity to learn both fundamental concepts in the field of
  computer science and explore advanced technologies that are commonly employed
  at a wide variety of companies. To explore and develop new ideas, you should
  regularly communicate with your team and/or the teaching assistants and
  tutors.

\item {\bf Hone your technical writing skills}. Computer science assignments
  require to you write technical documentation and descriptions of your
  experiences when completing each task. Take extra care to ensure that your
  writing is interesting and both grammatically and technically correct,
  remembering that computer scientists must effectively communicate and
  collaborate with their team members and the tutors, teaching assistants, and
  course instructor.

\item {\bf Review the Honor Code on the syllabus}. While you may discuss your
  assignments with others, copying source code or writing is a violation of
  Allegheny College's Honor Code.

\end{itemize}

\section*{Reading Assignment}

If you have not done so already, please read all of the relevant ``GitHub
Guides'', available at \url{https://guides.github.com/}, that explain how to use
many of the features that GitHub provides. In particular, please make sure that
you have read guides such as ``Mastering Markdown'' and ``Documenting Your
Projects on GitHub''; each of them will help you to understand how to use both
GitHub and GitHub Classroom. To do well on this assignment, you should also read
Chapter 1 in the textbook, paying particularly close attention to Code Fragments
1.5, 1.6, and 1.7. Please see the instructor or one of the teaching assistants
if you have questions on this reading assignment.

\section*{Configuring Git and GitHub}

During this laboratory assignment and the subsequent laboratory and practical
assignments, we will securely communicate with the GitHub servers that will host
all of the project templates and your submitted deliverables. In this
assignment, you will perform all of the steps to configure your account on
GitHub and you will start your first assignment using GitHub Classroom.
Throughout this assignment, you should refer to the following web sites for more
information: \url{https://guides.github.com/} and, in particular,
\url{https://guides.github.com/activities/hello-world/}. You can also learn more
about GitHub Classroom by visiting \url{https://classroom.github.com/}. As you
will be required to use Git, an industry standard tool, in all of the remaining
laboratory and practical assignments and during the class sessions, you should
keep a record of all of the steps that you complete and the challenges that you
face. Note that you do not need to complete these steps if you already have an
account on GitHub and you can already use your laptop to finish assignments.

\begin{enumerate}

  \itemsep 0in

  \item If you do not already have a GitHub account, then please go to the
    GitHub web site and create one, making sure that you use your
    \command{allegheny.edu} email address so that you can join GitHub as a
    student at an accredited educational institution. You are also encouraged to
    sign up for GitHub's ``Student Developer Pack'' at
    \url{https://education.github.com/pack}, qualifying you to receive free
    software development tools. Additionally, please add a description of
    yourself and an appropriate professional photograph to your GitHub profile.
    Unless your username is taken, you should also pick your GitHub username to
    be the same as Allegheny's Google-based email account. Now, in the
    \channel{#labs} channel of our Slack workspace, please type on one line your
    full name, \command{allegheny.edu} email address, and your new GitHub
    username. If you have already typed this information into a Slack channel,
    you do not need to do so again.

  \item If you have never done so before, you must use the \command{ssh-keygen}
    program to create secure-shell keys that you can use to support your
    communication with GitHub. But, to start, this task requires you to type
    commands in a program that is known as a terminal.
    %
    Your terminal program will vary depending on your operating system. For
    instance, if you are running Linux, you can click on an icon that contains
    the \command{>} symbol or press the ``Super'' key, start typing the word
    ``terminal'', and then select that program. Another way to open a terminal
    involves typing the key combination \command{<Ctrl>-<Alt>-t}.
    %
    On the Windows operating system you may want to use the ``Command Prompt''
    or the ``Power Shell'' and on MacOS you can use ``Terminal''.
    %
    Your terminal will display as a box into which you can type commands.
    %
    For the next step, you may need to separately install \command{ssh-keygen}
    if it is not on your laptop.

  \item Now that you have started the terminal, you will now need to type the
    \command{ssh-keygen} command in it. Follow the prompts to create your keys
    and save them in the default directory. That is, you should press ``Enter''
    after you are prompted to \command{Enter file in which to save the key ...
    :} and then type your selected passphrase whenever you are prompted to do
    so. Please note that a ``passphrase'' is like a password that you will type
    when you need to prove your identify to GitHub. What files does
    \command{ssh-keygen} produce? Where does this program store these files by
    default? Do you have any questions about completing this step?

  \item Once you have created your SSH keys, you should raise your hand to
    invite either a technical leader or the course instructor to help you with
    the next steps. First, you must log into GitHub and look in the right corner
    for an account avatar with a down arrow. Click on this link and then select
    the ``Settings'' option. Now, scroll down until you find the ``SSH and GPG
    keys'' label on the left, click to create a ``New SSH key'', and then upload
    your ssh key to GitHub. You can copy your SSH key to the clipboard by going
    to the terminal and typing ``{\tt cat \textasciitilde{}/.ssh/id\_rsa.pub}''
    command and then highlighting this output. When you are completing this step
    in your terminal window, please make sure that you only highlight the
    letters and numbers in your key---if you highlight any extra symbols or
    spaces then this step may not work correctly. Then, paste this into the
    GitHub text field in your web browser.

  \item Again, when you are completing these steps, please make sure that you
    take careful notes about the inputs, outputs, and behavior of each command.
    If there is something that you do not understand, then please ask the course
    instructor or the teaching assistant about it.

  \item Since this is your first laboratory assignment and you are still
    learning how to use the appropriate software, don't become frustrated if you
    make a mistake. Instead, use your mistakes as an opportunity for learning
    both about the necessary technology and the background and expertise of the
    other students in the class, the teaching assistants, and the course
    instructor. Remember, you can use Slack to talk with the instructor by
    typing \command{@gkapfham} in a channel.

\end{enumerate}

\section*{Accessing the Laboratory Assignment on GitHub}

To access the laboratory assignment, you should go into the
\channel{\#announcements} channel in our Slack team and find the announcement
that provides a link for it. Copy this link and paste it into a web browser.
Now, you should accept the assignment and see that GitHub Classroom created a
new GitHub repository for you to access the assignment's starting materials and
to store the completed version of your assignment. Specifically, to access your
new GitHub repository for this assignment, please click the green ``Accept''
button and then click the link that is prefaced with the label ``Your assignment
has been created here''. If you accepted the assignment and correctly followed
these steps, you should have created a GitHub repository with a name like
``Allegheny-Computer-Science-101-Fall-2019/computer-science-101-fall-2019-lab-1-gkapfham''.
Unless you provide the instructor with written documentation of the severe and
extenuating circumstances that you are facing, not accepting the assignment
means that you automatically receive a failing grade for it.

Before you move to the next step of this assignment, please make sure that you
read all of the content on the web site for your new GitHub repository, paying
close attention to the technical details about the commands that you will type
and the output that your program must produce. Now you are ready to download the
starting materials to your laboratory computer. Click the ``Clone or download''
button and, after ensuring that you have selected ``Clone with SSH'', please
copy this command to your clipboard. At this point, you can open a new terminal
window and type the command \command{mkdir cs101F2019}. To enter into this
directory you should now type \command{cd cs101F2019}. Next, you can type the
command \command{ls} and see that there are no files or directories inside of
this directory. By typing \command{git clone} in your terminal and then pasting
in the string that you copied from the GitHub site you will download all of the
code for this assignment. For instance, if the course instructor ran the
\command{git clone} command in the terminal, it would look like:

\begin{lstlisting}
  git clone git@github.com:Allegheny-Computer-Science-101-F2019/computer-science-101-fall-2019-lab-1-gkapfham.git
\end{lstlisting}

After this command finishes, you can use \command{cd} to change into the new
directory. If you want to \step{go back} one directory from your current
location, then you can type the command \command{cd ..}. Please continue to use
the \command{cd} and \command{ls} commands to explore the files that you
automatically downloaded from GitHub. What files and directories do you see?
What do you think is their purpose? Spend some time exploring, sharing your
discoveries with a neighbor and a \mbox{teaching assistant}.

\section*{Enhancing and Testing a Java Program}

In order view the source code of a Java program or a Markdown-based writing
assignment, you need a text editor. There are many different text editors that
are available and you should feel free to explore them on your own. For
instance, you may adopt either the VS Code or the Atom text editor to complete
this laboratory assignment. Please study the source code of the \mainprogram,
noticing that some of its methods are incomplete. You can refer to the Code
Fragments in the textbook to discover the missing source code and add it back
into the program. Next, you should also notice that the \mainprogram{} has a
test suite called \testprogram{} that is written in the JUnit testing framework;
please refer to Section 1.9 and the comments in the source code of this test
suite to learn more about how it works. Intuitively, a JUnit test suite has
individual test cases, written as Java methods, that each work to establish a
confidence in the correctness of \mainprogram{}.

% Introduce the use of Docker and the use of Gradle inside of Docker

The instructor will provide you with the necessary software development tools
through a Docker container. Once you have installed Docker Desktop, you can use
the following \command{docker run} command to start \command{gradle grade} as a
containerized application, using the ``DockaGator'' Docker image available on
DockerHub. You can run this command to run the \command{gradle grade} on your
project:

\vspace*{-.1in}
\begin{verbatim}
docker run --rm --name dockagator \
  -v "$(pwd)":/project \
  -v "$HOME/.dockagator":/root/.local/share \
  gatoreducator/dockagator
\end{verbatim}
\vspace*{-.05in}

The aforementioned command will use \program{"\$(pwd)"} (i.e., the current
directory) as the project directory and \program{"\$HOME/.dockagator"} as the
cached GatorGrader directory. Please note that both of these directories must
exist, although only the project directory must contain some content. Generally,
the project directory should contain the source code and technical writing for
this assignment, as provided to you through GitHub during the completion of a
previous step. Additionally, the cache directory should not contain anything
other than directories and programs created by DockaGator, thus ensuring that
they are not otherwise overwritten during the completion of the assignment. To
ensure that the previous command will work correctly, you should create the
cache directory by running the command \command{mkdir \$HOME/.dockagator}; you
will only need to do this once. If the above \command{docker run} command does
not work correctly on the Windows operating system, then you may need to instead
run the following command to work around limitations in the terminal window:

\vspace*{-.1in}
\begin{verbatim}
docker run --rm --name dockagator \
  -v "$(pwd):/project" \
  -v "$HOME/.dockagator:/root/.local/share" \
  gatoreducator/dockagator
\end{verbatim}
\vspace*{-.05in}

By adding \command{-it} before \command{--rm} and \command{/bin/bash} after
\command{gatoreducator/dockagator} you can enter into an interactive terminal
window in the Docker container. Now, if you want to \step{build} your program
you can type the command \gradlebuild{} in your terminal, thereby causing the
Java compiler to check your program for errors and get it ready to run. If you
get any error messages, go back to your text editor and figure out what you
mis-typed and fix it. Once you have solved the problem, make a note of the error
and the solution for resolving it. Re-save your program and then build it again
by re-running the \gradlebuild{}. If you cannot build \mainprogram{} in a Docker
container, then please talk with a technical leader or the instructor. You can
also run the test suite for the \mainprogram{} by typing \gradletest{} in the
terminal. What is the purpose of each of these test cases? What output do you
see from running the test suite? Does this output suggest that the program is
likely to produce the correct output? How did you know?

Once you resolve all of the building and testing errors, you can run your
program by typing \gradlerun{} in the terminal window---this is the ``execute''
step that will run your program and produce the designated output. You can see
if your program is producing the desired output by looking at Code Fragment 1.7.
Do you see that it produces twenty lines of output? Can you see that it computes
the correct numerical values? If not, then repair the program and re-build and
re-run it. Once the program runs, please reflect on this process. What step did
you find to be the most challenging? Why? You should write your reflections in a
file, called \reflection{}, that uses the Markdown writing language. You can
learn more about Markdown by viewing the aforementioned GitHub guide. To
complete this aspect of the assignment, you should write one high-quality
paragraph that reports on your experiences. Your \reflection{} should also
contain answers to other questions, about \mainprogram{}, posed at the end of
this assignment sheet.

% Since this is our first laboratory assignment and you are still learning how to
% use the appropriate hardware and software, don't become frustrated if you make a
% mistake. Instead, use your mistakes as an opportunity for learning both about
% the necessary technology and the background and expertise of the other students
% in the class, the technical leaders, and the course instructor.

\section*{Checking the Correctness of Your Program and Writing}

As verified by the Checkstyle tool, the source code for the \mainprogram{} and
\testprogram{} files must adhere to all of the requirements in the Google Java
Style Guide available at
\url{https://google.github.io/styleguide/javaguide.html}. The Markdown file that
contains your reflection must adhere to the standards described in the Markdown
Syntax Guide \url{https://guides.github.com/features/mastering-markdown/}.
Finally, your \reflection{} file should adhere to the Markdown standards
established by the \step{Markdown linting} tool available at
\url{https://github.com/markdownlint/markdownlint/} and the writing standards
set by the \step{prose linting} tool from \url{http://proselint.com/}. Instead
of requiring you to manually check that your deliverables adhere to these
industry-accepted standards, you will learn to use tools that make it easy for
you to automatically check if your submission meets the standards for
correctness. For instance, you have already seen that Gradle will run the
provided JUnit tests and alert you if they do not pass.

You can also use the GatorGrader tool to ensure that \mainprogram{} produces
exactly twenty lines of output and see that you committed to GitHub a sufficient
number of times when completing this assignment. To get started with the use of
GatorGrader, type the command \gatorgraderstart{} in your terminal window. If
your laboratory work does not meet all of the assignment's requirements, then
you will see a summary of the failing checks along with a statement giving the
percentage of checks that are currently passing. If you do have mistakes in your
assignment, then you will need to review GatorGrader's output, find the mistake,
and try to fix it, consulting your text book and course notes as needed. Once
your program is building correctly, fulfilling at least some of the assignment's
requirements, you should transfer your files to GitHub using the \gitcommit{}
and \gitpush{} commands. For example, if you want to signal that the
\mainprogramsource{} file has been changed and is ready for transfer to GitHub
you would first type \gitcommitmainprogram{} in your terminal, followed by
typing \gitpush{} and checking to see that the transfer to GitHub is successful.

% If you notice that transferring your code or writing to GitHub did not work
% correctly, then please try to determine why, asking a technical leader or the
% course instructor for assistance, if necessary.

After the course instructor enables \step{continuous integration} with a system
called Travis CI, when you use the \gitpush{} command to transfer your source
code to your GitHub repository, Travis CI will initialize a \step{build} of your
assignment, checking to see if it meets all of the requirements. If both your
source code and writing meet all of the established requirements, then you will
see a green \checkmark{} in the listing of commits in GitHub after awhile. If
your submission does not meet the requirements, a red \naughtmark{} will appear
instead. The instructor will reduce a student's grade for this assignment if the
red \naughtmark{} appears on the last commit in GitHub immediately before the
assignment's due date. Yet, if the green \checkmark{} appears on the last commit
in your GitHub repository, then you satisfied all of the main checks, thereby
allowing the course instructor to evaluate other aspects of your source code and
writing, as further described in the \step{Evaluation} section of this
assignment sheet. Unless you provide the instructor with documentation of the
extenuating circumstances that you are facing, no late work will be considered
towards your grade for this laboratory assignment.

Please note that if you are completing this project on your own laptop without
using Docker, then GatorGrader will only run correctly if you have installed all
of the programs on which it depends. You can see these dependencies (i.e.,
Gradle, Java, Markdownlint, Proselint, and Python) by reading the documentation
in your GitHub repository for this assignment. Note that assignment checking
will still work with Travis CI even if you do not have these programs installed
on your laptop. Please see the instructor or teaching assistant if you have
trouble installing these programs.

\section*{Summary of the Required Deliverables}

\noindent Students do not need to submit printed source code or technical
writing for any assignment in this course. Instead, this assignment invites you
to submit, using GitHub, the following deliverables.

\begin{enumerate}

  \setlength{\itemsep}{0in}

\item Stored in a Markdown file called \reflection{}, a multiple-paragraph
  response, each consisting of an adequate number words, for each prompt and
  questions, like those below.

  \vspace*{-.1in}

  \begin{enumerate}
    \itemsep 0em

    \item The {\tt CreditCard} constructor uses the {\tt this} keyword --- what
      does it mean?

    \item The {\tt CreditCard} class contains a {\tt static} method --- how does
      it work?

    \item The {\tt CreditCard} class has a {\tt main} method --- what is the
      purpose of this method?

    \item How does the {\tt CreditCard} class use a {\tt for} loop and a {\tt
      while} loop?

    \item The {\tt CreditCard} class creates a {\tt wallet} --- what is it and
      how does it work?

  \end{enumerate}

\vspace*{-.1in}

\item Stored in a Markdown file called \reflection{}, a one-paragraph (at least
  one hundred words) reflection on the challenges you faced and the steps that
  you took to solve them.

\item A properly documented, well-formatted, and correct version of
  \mainprogramsource{} that both meets all of the established requirements and
  produces the desired output.

\end{enumerate}

\section*{Evaluation of Your Laboratory Assignment}

Using a report that the instructor shares with you through the commit log in
GitHub, you will privately received a grade on this assignment and feedback on
your submitted deliverables. Your grade for the assignment will be a function of
the whether or not it was submitted in a timely fashion and if your program
received a green \checkmark{} indicating that it met all of the requirements.
Other factors will also influence your final grade on the assignment. In
addition to studying the efficiency and effectiveness and documentation of your
Java source code, the instructor will also evaluate the correctness of your
technical writing. If your submission receives a red \naughtmark{}, the
instructor will reduce your grade for the assignment. Finally, please remember
to read your GitHub repository's \program{README.md} file for a description of
the four grades that you will receive for this laboratory assignment.

\section*{Adhering to the Honor Code}

In adherence to the Honor Code, students should complete this assignment on an
individual basis. While it is appropriate for students in this class to have
high-level conversations about the assignment, it is necessary to distinguish
carefully between the student who discusses the principles underlying a problem
with others and the student who produces assignments that are identical to, or
merely variations on, someone else's work. Deliverables (e.g., Java source code
or Markdown-based technical writing) that are nearly identical to the work of
others will be taken as evidence of violating the \mbox{Honor Code}. Please see
the course instructor if you have questions about this policy.

\end{document}
