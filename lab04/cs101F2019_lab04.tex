\documentclass[11pt]{article}

% NOTE: The "Edit" sections are changed for each assignment

% Edit these commands for each assignment

\newcommand{\assignmentduedate}{October 10}
\newcommand{\assignmentassignedate}{October 3}
\newcommand{\assignmentnumber}{Four}

\newcommand{\labyear}{2019}
\newcommand{\labday}{Thursday}
\newcommand{\labtime}{2:30 pm}

\newcommand{\assigneddate}{Assigned: \labday, \assignmentassignedate, \labyear{} at \labtime{}}
\newcommand{\duedate}{Due: \labday, \assignmentduedate, \labyear{} at \labtime{}}

% Edit these commands to give the name to the main program

\newcommand{\mainprogram}{\lstinline{InsertionSort}}
\newcommand{\mainprogramsource}{\lstinline{src/main/java/labfour/InsertionSort.java}}

% Edit these commands to give the main program's output details

\newcommand{\mainprogramoutput}{four}

% Edit these commands to give the name to the test suite

\newcommand{\testprogram}{\lstinline{TestInsertionSort}}
\newcommand{\testprogramsource}{\lstinline{src/test/java/labfour/TestInsertionSort.java}}

% Edit this commands to describe key deliverables

\newcommand{\reflection}{\lstinline{writing/reflection.md}}

% Commands to describe key development tasks

% --> Running gatorgrader.sh
\newcommand{\gatorgraderstart}{\command{gradle grade}}
\newcommand{\gatorgradercheck}{\command{gradle grade}}

% --> Compiling and running and testing program with gradle
\newcommand{\gradlebuild}{\command{gradle build}}
\newcommand{\gradletest}{\command{gradle test}}
\newcommand{\gradlerun}{\command{gradle run}}

% Commands to describe key git tasks

% NOTE: Could be improved, problems due to nesting

\newcommand{\gitcommitfile}[1]{\command{git commit #1}}
\newcommand{\gitaddfile}[1]{\command{git add #1}}

\newcommand{\gitadd}{\command{git add}}
\newcommand{\gitcommit}{\command{git commit}}
\newcommand{\gitpush}{\command{git push}}
\newcommand{\gitpull}{\command{git pull}}

\newcommand{\gitcommitmainprogram}{\command{git commit src/main/java/labfour/InsertionSort.java -m "Your
descriptive commit message"}}

% Use this when displaying a new command

\newcommand{\command}[1]{``\lstinline{#1}''}
\newcommand{\program}[1]{\lstinline{#1}}
\newcommand{\url}[1]{\lstinline{#1}}
\newcommand{\channel}[1]{\lstinline{#1}}
\newcommand{\option}[1]{``{#1}''}
\newcommand{\step}[1]{``{#1}''}

\usepackage{pifont}
\newcommand{\checkmark}{\ding{51}}
\newcommand{\naughtmark}{\ding{55}}

\usepackage{listings}
\lstset{
  basicstyle=\small\ttfamily,
  columns=flexible,
  breaklines=true
}

\usepackage{fancyhdr}

\usepackage[margin=1in]{geometry}
\usepackage{fancyhdr}

\pagestyle{fancy}

\fancyhf{}
\rhead{Computer Science 101}
\lhead{Laboratory Assignment \assignmentnumber{}}
\rfoot{Page \thepage}
\lfoot{\duedate}

\usepackage{titlesec}
\titlespacing\section{0pt}{6pt plus 4pt minus 2pt}{4pt plus 2pt minus 2pt}

\newcommand{\labtitle}[1]
{
  \begin{center}
    \begin{center}
      \bf
      CMPSC 101\\Data Abstraction\\
      Fall 2019\\
      \medskip
    \end{center}
    \bf
    #1
  \end{center}
}

\begin{document}

\thispagestyle{empty}

\labtitle{Laboratory \assignmentnumber{} \\ \assigneddate{} \\ \duedate{}}

\section*{Objectives}

To continue to practice using GitHub to access the files for an assignment. You
will complete a programming project using source code provided in the textbook,
ultimately implementing and testing a complete solution for the sorting arrays
containing two primitive types. You will also continue to learn how to implement
and test a Java program and to write a Markdown file, practicing how to use an
automated grading tool to assess your progress towards completing the project.
Finally, you will practice using Docker containers to run programs like Gradle
and GatorGrader.

\section*{Suggestions for Success}

\begin{itemize}
  \setlength{\itemsep}{0pt}

\item {\bf Follow each step carefully}. Slowly read each sentence in the
  assignment sheet, making sure that you precisely follow each instruction. Take
  notes about each step that you attempt, recording your questions and ideas and
  the challenges that you faced. If you are stuck, then please tell a technical
  leader or instructor what assignment step you recently completed.

\item {\bf Regularly ask and answer questions}. Please log into Slack at the
  start of a laboratory or practical session and then join the appropriate
  channel. If you have a question about one of the steps in an assignment, then
  you can post it to the designated channel. Or, you can ask a student sitting
  next to you or talk with a technical leader or the course instructor.

\item {\bf Store your files in GitHub}. As in the previous laboratory
  assignments, you will be responsible for storing all of your files (e.g., Java
  source code and Markdown-based writing) in a Git repository using GitHub
  Classroom. Please verify that you have saved your source code in your Git
  repository by using \command{git status} to ensure that everything is updated.
  You can see if your assignment submission meets the established correctness
  requirements by using the provided checking tools on your local computer and
  in checking the commits in GitHub.

\item {\bf Keep all of your files}. Don't delete your programs, output files,
  and written reports after you submit them through GitHub; you will need them
  again when you study for the quizzes and examinations and work on the other
  laboratory, practical, and final project assignments.

\item {\bf Explore teamwork and technologies}. While certain aspects of the
  laboratory assignments will be challenging for you, each part is designed to
  give you the opportunity to learn both fundamental concepts in the field of
  computer science and explore advanced technologies that are commonly employed
  at a wide variety of companies. To explore and develop new ideas, you should
  regularly communicate with your team and/or the technical leaders.

\item {\bf Hone your technical writing skills}. Computer science assignments
  require to you write technical documentation and descriptions of your
  experiences when completing each task. Take extra care to ensure that your
  writing is interesting and both grammatically and technically correct,
  remembering that computer scientists must effectively communicate and
  collaborate with their team members and the student technical leaders and
  course instructor.

\item {\bf Review the Honor Code on the syllabus}. While you may discuss your
  assignments with others, copying source code or writing is a violation of
  Allegheny College's Honor Code.

\end{itemize}

\section*{Reading Assignment}

If you have not done so already, please read all of the relevant ``GitHub
Guides'', available at \url{https://guides.github.com/}, that explain how to
use many of the features that GitHub provides. In particular, please make sure
that you have read guides such as ``Mastering Markdown'' and ``Documenting Your
Projects on GitHub''; each of them will help you to understand how to use both
GitHub and GitHub Classroom. To do well on this assignment, you should also
read Chapter 3 in the textbook, paying particularly close attention to Sections
3.1 and 3.6. Please see the course instructor or one of the teaching assistants
if you have questions about this reading assignment.

\section*{Accessing the Laboratory Assignment on GitHub}

To access the laboratory assignment, you should go into the
\channel{\#announcements} channel in our Slack workspace and find the
announcement that provides a link for it. Copy this link and paste it into a
web browser. Now, you should accept the assignment and see that GitHub
Classroom created a new GitHub repository for you to access the assignment's
starting materials and to store the completed version of your assignment.
Specifically, to access your new GitHub repository for this assignment, please
click the green ``Accept'' button and then click the link that is prefaced with
the label ``Your assignment has been created here''. If you accepted the
assignment and correctly followed these steps, you should have created a GitHub
repository with a name like
``Allegheny-Computer-Science-101-Fall-2019/computer-science-101-fall-2019-lab-4-gkapfham''.
Unless you provide the instructor with documentation of the extenuating
circumstances that you are facing, not accepting the assignment means that you
automatically receive a failing grade for it.

Before you move to the next step of this assignment, please make sure that you
read all of the content on the web site for your new GitHub repository, paying
close attention to the technical details about the commands that you will type
and the output that your program must produce. Now you are ready to download the
starting materials to your laboratory computer. Click the ``Clone or download''
button and, after ensuring that you have selected ``Clone with SSH'', please
copy this command to your clipboard. To enter into your course directory you
should now type \command{cd cs101F2019}. By typing \command{git clone} in your
terminal and then pasting in the string that you copied from the GitHub site you
will download all of the code for this assignment. For instance, if the course
instructor ran the \command{git clone} command in the terminal, it would look
like:

\begin{lstlisting}
  git clone git@github.com:Allegheny-Computer-Science-101-F2019/computer-science-101-fall-2019-lab-4-gkapfham.git
\end{lstlisting}

After this command finishes, you can use \command{cd} to change into the new
directory. If you want to \step{go back} one directory from your current
location, then you can type the command \command{cd ..}. Please continue to use
the \command{cd} and \command{ls} commands to explore the files that you
automatically downloaded from GitHub. What files and directories do you see?
What do you think is their purpose? Spend some time exploring, sharing your
discoveries with a neighbor and a \mbox{student technical leader}.

\section*{Implementing and Testing an Array Sorting Program}

Please use a text editor to study the source code of the \mainprogram{} class,
noticing that some of its methods are incomplete. Now, if you want to
\step{build} your program you can type the command \gradlebuild{} in your
terminal, thereby causing the Java compiler to check your program for errors and
get it ready to run. However, once you ensure that the program correctly
compiles, you will see that it does not produce the output in
Figure~\ref{fig:output} and that many of the tests do not pass. What code is
missing in the provided files?
%
Ultimately, you will need to implement source code in both of the Java files
wherever you see the \command{TODO} label. For instance, the \testprogram{}
requires you to implement two \program{isSorted} ``helper'' methods that
determine if an array is sorted.

Next, you must implement an \program{InsertionSort.sort} method that follows the
outline in Code Fragment 3.6 to sort an array of either type \program{int[]} or
\program{char[]}. It is important to note that these methods should use the
\program{System.arraycopy} method to construct a ``shallow copy'' of the array
that is input to the \program{sort} method. Please refer to Section 3.6 and the
referenced JavaDoc documentation to learn more about how this method works.
%
Make sure that you understand how the use of a ``shallow copy'' would influence
arrays containing either primitives or references in a different fashion!
%
As you complete this laboratory assignment, your goal is to proceed
incrementally, adding the required code and then checking to see if an
additional test case passes. Once a new Java method is working, please make sure
that you commit the code to your repository with a descriptive message.

Once the program runs and the tests pass, please reflect on this process. What
step did you find to be the most challenging? Why? You should write your
reflections in a file, called \reflection{}, that uses the Markdown writing
language. You can learn more about Markdown by viewing the aforementioned GitHub
guide. To complete this aspect of the assignment, you should write one
high-quality paragraph that reports on your experiences. Your \reflection{} file
should also contain answers to other questions about \mainprogram{}'s
implementation and testing. For instance, can you explain how the
\testprogramsource{} uses the \program{isSorted} method? Can you also explain
why the some of the tests in the JUnit suite use a \program{for} loop? Finally,
can you say what characters and numbers are stored in the arrays used for
testing?

\begin{figure}[t]
  \centering
  \begin{verbatim}
  Before: [C, E, B, D, A, I, J, L, K, H, G, F]
  After : [A, B, C, D, E, F, G, H, I, J, K, L]
  Before: [1, 2, 4, 4, 9, 10, -10, 3, 8, 7, 20, 0]
  After : [-10, 0, 1, 2, 3, 4, 4, 7, 8, 9, 10, 20]
  \end{verbatim}
  \vspace*{-.35in}
  \caption{The Expected Output of the \mainprogram{} Program.}~\label{fig:output}
  \vspace*{-.25in}
\end{figure}

\section*{Checking the Correctness of Your Program and Writing}

As verified by the Checkstyle tool, the source code for the \mainprogram{} and
the other Java files must adhere to all of the requirements in the Google Java
Style Guide available at
\url{https://google.github.io/styleguide/javaguide.html}. The Markdown file that
contains your reflection must adhere to the standards described in the Markdown
Syntax Guide at \url{https://guides.github.com/features/mastering-markdown/}.
Finally, your \reflection{} file should adhere to the Markdown standards
established by the \step{Markdown linting} tool available at
\url{https://github.com/markdownlint/markdownlint/}. Instead of requiring you to
manually check that your deliverables adhere to these industry-accepted
standards, GatorGrader makes it easy for you to automatically check if your
submission meets the standards for correctness. For instance, automated tools
will run the provided JUnit tests, check to ensure that \mainprogram{} produces
\mainprogramoutput{} lines of output, and see that you commit to GitHub a
sufficient number of times when completing this assignment.

To get started with the use of GatorGrader, type the command
\gatorgraderstart{} in your terminal window. If your laboratory work does not
meet all of the assignment's requirements, then you will see a summary of the
failing checks along with a statement giving the percentage of checks that are
currently passing. If you do have mistakes in your assignment, then you will
need to review GatorGrader's output, find the mistake, and try to fix it,
consulting your text book and course notes as needed. Once your program is
building correctly, fulfilling at least some of the assignment's requirements,
you should transfer your files to GitHub using the \gitcommit{} and \gitpush{}
commands. For example, if you want to signal that the \mainprogramsource{} file
has been changed and is ready for transfer to GitHub you would first type
\gitcommitmainprogram{} in your terminal, followed by typing \gitpush{} and
checking to see that the transfer to GitHub is successful.

\begin{figure}[t]
  \centering
  \begin{verbatim}
    labfour.TestInsertionSort > testisSortedCheckerWorksForSortedChar PASSED
    labfour.TestInsertionSort > testisSortedCheckerWorksForUnSortedChar PASSED
    labfour.TestInsertionSort > testisSortedCheckerWorksForSortedInt PASSED
    labfour.TestInsertionSort > testisSortedCheckerWorksForUnSortedInt PASSED
    labfour.TestInsertionSort > testInsertionSortWithChar PASSED
    labfour.TestInsertionSort > testInsertionSortWithInt PASSED
    labfour.TestInsertionSort > testInsertionSortWithManyOrderedInts PASSED
    labfour.TestInsertionSort > testInsertionSortWithManyOrderedChar PASSED
    labfour.TestInsertionSort > testInsertionSortWithManyReversedInts PASSED
    labfour.TestInsertionSort > testInsertionSortWithManyReversedChars PASSED
    labfour.TestInsertionSort > testInsertionSortWithManyRandomInts PASSED
    labfour.TestInsertionSort > testInsertionSortWithManyRandomChars PASSED
  \end{verbatim}
  \vspace*{-.35in}
  \caption{The Expected Output of the \testprogram{} Test Suite.}~\label{fig:test}
  \vspace*{-.25in}
\end{figure}

After the course instructor enables \step{continuous integration} with a system
called Travis CI, when you use the \gitpush{} command to transfer your source
code to your GitHub repository, Travis CI will initialize a \step{build} of your
assignment, checking to see if it meets all of the requirements. If both your
source code and writing meet all of the established requirements, then you will
see a green \checkmark{} in the listing of commits in GitHub after awhile. If
your submission does not meet the requirements, a red \naughtmark{} will appear
instead. The instructor will reduce a student's grade for this assignment if the
red \naughtmark{} appears on the last commit in GitHub immediately before the
assignment's due date. Yet, if the green \checkmark{} appears on the last commit
in your GitHub repository, then you satisfied all of the main checks, thereby
allowing the course instructor to evaluate other aspects of your source code and
writing, as further described in the \step{Evaluation} section of this
assignment sheet. Unless you provide the instructor with documentation of the
extenuating circumstances that you are facing, no late work will be considered
towards your grade for this laboratory assignment.

\section*{Summary of the Required Deliverables}

\noindent Students do not need to submit printed source code or technical
writing for any assignment in this course. Instead, this assignment invites you
to submit, using GitHub, the following deliverables.

\begin{enumerate}

  \setlength{\itemsep}{0in}

\item Stored in a Markdown file called \reflection{}, a multiple-paragraph
  response across the stated questions and fenced code blocks for the output of
  the program and the test suite.

\item A properly documented and correct version of the two Java source files
  (i.e., sorting methods and test suite) that meet all of the established
  requirements and produce the desired output.
  %
  Make sure that you add the source code that fully tests and performs arrays
  sorting.

\end{enumerate}

\section*{Evaluation of Your Laboratory Assignment}

Using a report that the instructor shares with you through the commit log in
GitHub, you will privately received a grade on this assignment and feedback on
your submitted deliverables. Your grade for the assignment will be a function of
the whether or not it was submitted in a timely fashion and if your program
received a green \checkmark{} indicating that it met all of the requirements.
Other factors will also influence your final grade on the assignment. In
addition to studying the efficiency and effectiveness and documentation of your
Java source code, the instructor will also evaluate the correctness of your
technical writing. If your submission receives a red \naughtmark{}, the
instructor will reduce your grade for the assignment. Finally, please remember
to read your GitHub repository's \program{README.md} file for a description of
the four grades that you will receive for this laboratory assignment.

% \section*{Adhering to the Honor Code}

% In adherence to the Honor Code, students should complete this assignment on an
% individual basis. While it is appropriate for students in this class to have
% high-level conversations about the assignment, it is necessary to distinguish
% carefully between the student who discusses the principles underlying a
% problem with others and the student who produces assignments that are
% identical to, or merely variations on, someone else's work. Deliverables
% (e.g., Java source code or Markdown-based technical writing) that are nearly
% identical to the work of others will be taken as evidence of violating the
% \mbox{Honor Code}. Please see the course instructor if you have questions
% about this policy.

\end{document}
