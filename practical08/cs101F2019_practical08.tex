\documentclass[11pt]{article}

% NOTE: The "Edit" sections are changed for each assignment

% Edit these commands for each assignment

\newcommand{\assignmentduedate}{November 18}
\newcommand{\assignmentassignedate}{November 15}
\newcommand{\assignmentnumber}{Seven}

\newcommand{\labyear}{2019}
\newcommand{\labdueday}{Monday}
\newcommand{\labassignday}{Friday}
\newcommand{\labtime}{9:00 am}

\newcommand{\assigneddate}{Assigned: \labassignday, \assignmentassignedate, \labyear{} at \labtime{}}
\newcommand{\duedate}{Due: \labdueday, \assignmentduedate, \labyear{} at \labtime{}}

% Edit these commands to give the name to the main program

\newcommand{\mainprogram}{\lstinline{Experiment}}
\newcommand{\mainprogramsource}{\lstinline{src/main/java/practicalseven/experiment/Experiment.java}}

% Edit these commands to give the main program's output details

\newcommand{\mainprogramoutput}{five}

% Edit these commands to give the name to the test suite

\newcommand{\testprogram}{\lstinline{TestExperiment}}
\newcommand{\testprogramsource}{\lstinline{src/test/java/practicalseven/TestExperiment.java}}

% Edit this commands to describe key deliverables

\newcommand{\reflection}{\lstinline{writing/reflection.md}}

% Commands to describe key development tasks

% --> Running gatorgrader
\newcommand{\gatorgraderstart}{\command{gradle grade}}
\newcommand{\gatorgradercheck}{\command{gradle grade}}

% --> Compiling and running and testing program with gradle
\newcommand{\gradlebuild}{\command{gradle build}}
\newcommand{\gradletest}{\command{gradle test}}
\newcommand{\gradlerun}{\command{gradle run}}

% Commands to describe key git tasks

% NOTE: Could be improved, problems due to nesting

\newcommand{\gitcommitfile}[1]{\command{git commit #1}}
\newcommand{\gitaddfile}[1]{\command{git add #1}}

\newcommand{\gitadd}{\command{git add}}
\newcommand{\gitcommit}{\command{git commit}}
\newcommand{\gitpush}{\command{git push}}
\newcommand{\gitpull}{\command{git pull}}

\newcommand{\gitcommitmainprogram}{\command{git commit src/main/java/practicalseven/experiment/Experiment.java -m "Your
descriptive commit message"}}

% Use this when displaying a new command

\newcommand{\command}[1]{``\lstinline{#1}''}
\newcommand{\program}[1]{\lstinline{#1}}
\newcommand{\url}[1]{\lstinline{#1}}
\newcommand{\channel}[1]{\lstinline{#1}}
\newcommand{\option}[1]{``{#1}''}
\newcommand{\step}[1]{``{#1}''}

\usepackage{pifont}
\newcommand{\checkmark}{\ding{51}}
\newcommand{\naughtmark}{\ding{55}}

\usepackage{listings}
\lstset{
  basicstyle=\small\ttfamily,
  columns=flexible,
  breaklines=true
}

\usepackage{fancyhdr}

\usepackage[margin=1in]{geometry}
\usepackage{fancyhdr}

\pagestyle{fancy}

\fancyhf{}
\rhead{Computer Science 101}
\lhead{Practical Assignment \assignmentnumber{}}
\rfoot{Page \thepage}
\lfoot{\duedate}

\usepackage{titlesec}
\titlespacing\section{0pt}{6pt plus 4pt minus 2pt}{4pt plus 2pt minus 2pt}

\newcommand{\labtitle}[1]
{
  \begin{center}
    \begin{center}
      \bf
      CMPSC 101\\Data Abstraction\\
      Fall 2019\\
      \medskip
    \end{center}
    \bf
    #1
  \end{center}
}

\begin{document}

\thispagestyle{empty}

\labtitle{Practical \assignmentnumber{} \\ \assigneddate{} \\ \duedate{}}

\section*{Objectives}

To continue to practice using GitHub to access the files for an assignment. You
will complete a programming project using source code provided in the textbook
and by the course instructor ultimately implementing a framework for
experimentally evaluating three sorting algorithms. You will also practice
creating a data table and calculating an order of growth ratio. You will also
continue to learn how to implement and test a Java program and detect
logarithmic growth, practicing how to use an automated grading tool to assess
your progress towards correctly completing the project.

\section*{Suggestions for Success}

\begin{itemize}
  \setlength{\itemsep}{0pt}

\item {\bf Follow each step carefully}. Slowly read each sentence in the
  assignment sheet, making sure that you precisely follow each instruction. Take
  notes about each step that you attempt, recording your questions and ideas and
  the challenges that you faced. If you are stuck, then please tell a technical
  leader or instructor what assignment step you recently completed.

\item {\bf Regularly ask and answer questions}. Please log into Slack at the
  start of a laboratory or practical session and then join the appropriate
  channel. If you have a question about one of the steps in an assignment, then
  you can post it to the designated channel. Or, you can ask a student sitting
  next to you or talk with a technical leader or the course instructor.

\item {\bf Store your files in GitHub}. Starting with this laboratory
  assignment, you will be responsible for storing all of your files (e.g., Java
  source code and Markdown-based writing) in a Git repository using GitHub
  Classroom. Please verify that you have saved your source code in your Git
  repository by using \command{git status} to ensure that everything is updated.
  You can see if your assignment submission meets the established correctness
  requirements by using the provided checking tools on your local computer and
  in checking the commits in GitHub.

\item {\bf Keep all of your files}. Don't delete your programs, output files,
  and written reports after you submit them through GitHub; you will need them
  again when you study for the quizzes and examinations and work on the other
  laboratory, practical, and final project assignments.

\item {\bf Explore teamwork and technologies}. While certain aspects of the
  laboratory assignments will be challenging for you, each part is designed to
  give you the opportunity to learn both fundamental concepts in the field of
  computer science and explore advanced technologies that are commonly employed
  at a wide variety of companies. To explore and develop new ideas, you should
  regularly communicate with your team and/or the student technical leaders.

\item {\bf Hone your technical writing skills}. Computer science assignments
  require to you write source code documentation and detailed notes of your
  experiences when completing each task. Take extra care to ensure that your
  writing is interesting and both grammatically and technically correct,
  remembering that computer scientists must effectively communicate and
  collaborate with their team members and the technical leaders and course
  instructor.

\item {\bf Review the Honor Code on the syllabus}. While you may discuss your
  assignments with others, copying source code or writing is a violation of
  Allegheny College's Honor Code.

\end{itemize}

\section*{Reading Assignment}

If you have not done so already, please read all of the relevant ``GitHub
Guides'', available at \url{https://guides.github.com/}, that explain how to use
many of the features that GitHub provides. In particular, please make sure that
you have read guides such as ``Mastering Markdown'' and ``Documenting Your
Projects on GitHub''; each of them will help you to understand how to use both
GitHub and GitHub Classroom. You should also read Section 4.1 through 4.3,
focusing on Table 4.2 and Figure 4.4 and the comments about algorithm analysis
on pages 153 and 172. Please see the course instructor or a technical leader if
you have questions about this reading assignment.

\section*{Accessing the Practical Assignment on GitHub}

To access the practical assignment, you should go into the
\channel{\#announcements} channel in our Slack team and find the announcement
that provides a link for it. Copy this link and paste it into a web browser.
Now, you should accept the assignment and see that GitHub Classroom created a
new GitHub repository for you to access the assignment's starting materials and
to store the completed version of your assignment. Specifically, to access your
new GitHub repository for this assignment, please click the green ``Accept''
button and then click the link that is prefaced with the label ``Your assignment
has been created here''. If you accepted the assignment and correctly followed
these steps, you should have created a GitHub repository with a name like
``Allegheny-Computer-Science-101-Fall-2019/computer-science-101-fall-2019-practical-7-gkapfham''.
Unless you provide the instructor with documentation of the extenuating
circumstances that you are facing, not accepting the practical assignment means
that you automatically receive a failing grade for it.

Before you move to the next step of this assignment, please make sure that you
read all of the content on the web site for your new GitHub repository, paying
close attention to the technical details about the commands that you will type
and the output that your program must produce. Now you are ready to download the
starting materials to your laboratory computer. Click the ``Clone or download''
button and, after ensuring that you have selected ``Clone with SSH'', please
copy this command to your clipboard. To enter into your course directory you
should now type \command{cd cs101F2019}. By typing \command{git clone} in your
terminal and then pasting in the string that you copied from the GitHub site you
will download all of the code for this assignment. For instance, if the course
instructor ran the \command{git clone} command in the terminal, it would look
like:

\begin{lstlisting}
  git clone git@github.com:Allegheny-Computer-Science-101-F2019/computer-science-101-fall-2019-practical-7-gkapfham.git
\end{lstlisting}

After this command finishes, you can use \command{cd} to change into the new
directory. If you want to \step{go back} one directory from your current
location, then you can type the command \command{cd ..}. Please continue to use
the \command{cd} and \command{ls} commands to explore the files that you
automatically downloaded from GitHub. What files and directories do you see?
What do you think is their purpose? Spend some time exploring, sharing your
discoveries with a neighbor and a \mbox{teaching assistant}. Specifically, each
student should draw a diagram to show the relationship between the Java classes
provided for this practical assignment. Make sure that you have the instructor
check your project diagram.

\section*{Comparing the Growth Rates of Three Sorting Algorithms}

There are several \command{TODO} markers inside of the provided source code.
Ultimately, it is your responsibility to read each of these and provide the
required source code. With that said, the remainder of this section will provide
some pointers to help you to implement the needed code. This assignment asks you
to run experiments that produce output like that which Figure~\ref{fig:output}
provides. This figure shows that now you must conduct an experiment with the
\program{QuickSort} algorithm provided by the \program{Arrays.sort} method.
Before you start calling this method, please take time to read the JavaDoc
documentation for it. What sorting algorithm does \program{Arrays.sort} use?
What is the worst-case time complexity for this method? Why is this approach to
sorting faster than a version that does not use these ``pivots''? If you have
questions about this method, then please ask the instructor. Now, you are ready
to implement methods like \program{public char[] sort(char[] source)} for
\program{QuickSort}. Also, don't forget to add the required source code for the
\program{BubbleSort} algorithm. Before you start to run experiments, please make
sure that all of the tests pass for all of the sorting algorithms! Finally, if
you have extra time, please consider extending the benchmarking framework so
that it supports different type of input arrays (e.g., reversed arrays or arrays
with data that is already sorted).

It is worth pointing out that your textbook contains several useful insights
into the pattern that you should follow when thinking about the running time of
the three sorting algorithms. For instance, when describing the results from
running an experiment, page 153 notes that ``[A]s the value of $n$ is doubled,
the running time of {\tt repeat1} typically increases more than fourfold.'' What
does this suggest about the likely ``worst-case time complexity'' of the {\tt
repeat1} method? How can you apply this intuition to analyze the results that
you collect when running an experiment? Additionally, page 172 includes the
following statement when describing the performance of {\tt repeat2}: ``the
running times in that table $\ldots$ demonstrate a trend of approximately
doubling each time the problem size doubles.'' Again, what would this
observation suggest about the likely worst-case time complexity of {\tt
repeat2}? Now, what does the data table in Figure~\ref{fig:output} tell us about
the likely worst-case time complexity of \program{QuickSort}? Do you have any
questions about this thought process? Please talk with the course instructor if
you do not understand the data table in Figure~\ref{fig:output}.
%
Finally, for your reference, here is a portion of the source code for the
\program{BubbleSort} algorithm. Can you integrate this source code into the
provided method? Where should the call to \program{System.arraycopy(source,
ARRAY_START, sorted, ARRAY_START, source.length);} occur in this method? Does
the source code of this method suggest anything about the worst-case time
complexity of the \program{BubbleSort} algorithm?

\begin{verbatim}
    for (int i = 0; i < length; i++) {
      for (int j = 0; j < (length - 1); j++) {
        if (sorted[j] > sorted[j + 1]) {
          int temporary = sorted[j];
          sorted[j] = sorted[j + 1];
          sorted[j + 1] = temporary;
        }
      }
    }
\end{verbatim}

\begin{figure}[t]
  \centering
  \begin{verbatim}
    Results of an experiment campaign with QuickSort:

    Size (#)        Timing (ms)     Ratio (#)
    250             1               0
    500             0               0
    1000            1               0
    2000            1               1
    4000            2               2
    8000            1               1
    16000           3               3
    32000           5               2
    64000           14              3
    128000          27              2
  \end{verbatim}

  \vspace*{-.35in}
  \caption{A Portion of the Expected Output of the \mainprogram{} Program.}~\label{fig:output}
  \vspace*{-.25in}
\end{figure}

\section*{Checking the Correctness of Your Program and Writing}

As verified by Checkstyle tool, the source code for \mainprogram{} and its test
suite must adhere to all of the requirements in the Google Java Style Guide
available at \url{https://google.github.io/styleguide/javaguide.html}. Instead
of requiring you to manually check that your deliverables adhere to these
industry-accepted standards, tools will make it easy for you to automatically
check if your submission meets the standards for correctness. For instance,
Gradle will run the provided tests, while GatorGrader will check to ensure that
\mainprogram{} uses the correct number of \command{println} statements and that
you committed to GitHub a sufficient number of times during this assignment. To
get started with the use of GatorGrader, type the command \gatorgraderstart{}
in your terminal window. If you have mistakes in your assignment, then you will
need to review GatorGrader's output, find the mistake, and try to fix it. Once
your program is building correctly, fulfilling at least some of the
assignment's requirements, you should transfer your files to GitHub using the
\gitcommit{} and \gitpush{} commands. For example, if you want to signal that
the \mainprogramsource{} file has been changed and is ready for transfer to
GitHub you would first type \gitcommitmainprogram{} in your terminal, followed
by typing \gitpush{}.

After the course instructor enables \step{continuous integration} with a system
called Travis CI, when you use the \gitpush{} command to transfer your source
code to your GitHub repository, Travis CI will initialize a \step{build} of your
assignment, checking to see if it meets all of the requirements. If both your
source code and writing meet all of the established requirements, then you will
see a green \checkmark{} in the listing of commits in GitHub after awhile. If
your submission does not meet the requirements, a red \naughtmark{} will appear
instead. The instructor will reduce a student's grade for this assignment if the
red \naughtmark{} appears on the last commit in GitHub immediately before the
assignment's due date. Yet, if the green \checkmark{} appears on the last commit
in your GitHub repository, then you satisfied all of the main checks, thereby
allowing the course instructor to evaluate other aspects of your source code and
writing, as further described in the \step{Evaluation} section of this
assignment sheet. Unless you provide the instructor with documentation of the
extenuating circumstances that you are facing, no late work will be considered
towards your grade for this practical assignment.

\section*{Summary of the Required Deliverables}

\noindent Students do not need to submit printed source code or technical
writing for any assignment in this course. Instead, this assignment invites you
to submit, using GitHub, the following deliverables.

\begin{enumerate}

  \setlength{\itemsep}{0in}

\item A properly documented and correct version of all the Java files (e.g.,
  three sorting algorithms, experiments, and tests) that meets the requirements
  and produces the desired output.

\end{enumerate}

\section*{Evaluation of Your Practical Assignment}

Using a report that the instructor shares with you through the commit log in
GitHub, you will privately received a grade on this assignment and feedback on
your submitted deliverables. Your grade for the assignment will be a function of
the whether or not it was submitted in a timely fashion and if your program
received a green \checkmark{} indicating that it met all of the baseline
requirements checked by GatorGrader.
%
Please remember to read your GitHub repository's \program{README.md} file for a
description of the completion grade that you will receive for this practical
assignment.
%
You can talk with the instructor if you have questions about the evaluation of
this practical assignment.

\end{document}
