\documentclass[11pt]{article}

% NOTE: The "Edit" sections are changed for each assignment

% Edit these commands for each assignment

\newcommand{\assignmentduedate}{October 3}
\newcommand{\assignmentassignedate}{September 26}
\newcommand{\assignmentnumber}{Three}

\newcommand{\labyear}{2019}
\newcommand{\labday}{Thursday}
\newcommand{\labtime}{2:30 pm}

\newcommand{\assigneddate}{Assigned: \labday, \assignmentassignedate, \labyear{} at \labtime{}}
\newcommand{\duedate}{Due: \labday, \assignmentduedate, \labyear{} at \labtime{}}

% Edit these commands to give the name to the main program

\newcommand{\mainprogram}{\lstinline{Reverser}}
\newcommand{\mainprogramsource}{\lstinline{src/main/java/labthree/Reverser.java}}

\newcommand{\mainprogramhelper}{\lstinline{Sentences}}
\newcommand{\mainprogramhelpersource}{\lstinline{src/main/java/labthree/Sentences.java}}

% Edit these commands to give the main program's output details

\newcommand{\mainprogramoutput}{twenty-eight}

% Edit these commands to give the name to the test suite

\newcommand{\testprogram}{\lstinline{TestReverser}}
\newcommand{\testprogramsource}{\lstinline{src/test/java/labthree/TestReverser.java}}

% Edit this commands to describe key deliverables

\newcommand{\reflection}{\lstinline{writing/reflection.md}}

% Commands to describe key development tasks

% --> Running gatorgrader.sh
\newcommand{\gatorgraderstart}{\command{gradle grade}}
\newcommand{\gatorgradercheck}{\command{gradle grade}}

% --> Compiling and running and testing program with gradle
\newcommand{\gradlebuild}{\command{gradle build}}
\newcommand{\gradletest}{\command{gradle test}}
\newcommand{\gradlerun}{\command{gradle run}}

% Commands to describe key git tasks

% NOTE: Could be improved, problems due to nesting

\newcommand{\gitcommitfile}[1]{\command{git commit #1}}
\newcommand{\gitaddfile}[1]{\command{git add #1}}

\newcommand{\gitadd}{\command{git add}}
\newcommand{\gitcommit}{\command{git commit}}
\newcommand{\gitpush}{\command{git push}}
\newcommand{\gitpull}{\command{git pull}}

\newcommand{\gitcommitmainprogram}{\command{git commit src/main/java/labthree/Reverser.java -m "Your
descriptive commit message"}}

% Use this when displaying a new command

\newcommand{\command}[1]{``\lstinline{#1}''}
\newcommand{\program}[1]{\lstinline{#1}}
\newcommand{\url}[1]{\lstinline{#1}}
\newcommand{\channel}[1]{\lstinline{#1}}
\newcommand{\option}[1]{``{#1}''}
\newcommand{\step}[1]{``{#1}''}

\usepackage{pifont}
\newcommand{\checkmark}{\ding{51}}
\newcommand{\naughtmark}{\ding{55}}

\usepackage{listings}
\lstset{
  basicstyle=\small\ttfamily,
  columns=flexible,
  breaklines=true
}

\usepackage{fancyhdr}

\usepackage[margin=1in]{geometry}
\usepackage{fancyhdr}

\pagestyle{fancy}

\fancyhf{}
\rhead{Computer Science 101}
\lhead{Laboratory Assignment \assignmentnumber{}}
\rfoot{Page \thepage}
\lfoot{\duedate}

\usepackage{titlesec}
\titlespacing\section{0pt}{6pt plus 4pt minus 2pt}{4pt plus 2pt minus 2pt}

\newcommand{\labtitle}[1]
{
  \begin{center}
    \begin{center}
      \bf
      CMPSC 101\\Data Abstraction\\
      Fall 2019\\
      \medskip
    \end{center}
    \bf
    #1
  \end{center}
}

\begin{document}

\thispagestyle{empty}

\labtitle{Laboratory \assignmentnumber{} \\ \assigneddate{} \\ \duedate{}}

\section*{Objectives}

To continue to practice using GitHub to access the files for an assignment. You
will complete a programming project using source code provided in the textbook,
ultimately implementing and testing a generic solution for reversing arrays of
any reference data type. You will also continue to learn how to implement and
test a Java program and to write a Markdown file, practicing how to use an
automated grading tool to assess your progress towards correctly completing the
project.

\section*{Suggestions for Success}

\begin{itemize}
  \setlength{\itemsep}{0pt}

\item {\bf Follow each step carefully}. Slowly read each sentence in the
  assignment sheet, making sure that you precisely follow each instruction. Take
  notes about each step that you attempt, recording your questions and ideas and
  the challenges that you faced. If you are stuck, then please tell a technical
  leader or instructor what assignment step you recently completed.

\item {\bf Regularly ask and answer questions}. Please log into Slack at the
  start of a laboratory or practical session and then join the appropriate
  channel. If you have a question about one of the steps in an assignment, then
  you can post it to the designated channel. Or, you can ask a student sitting
  next to you or talk with a technical leader or the course instructor.

\item {\bf Store your files in GitHub}. As in the previous laboratory
  assignments, you will be responsible for storing all of your files (e.g., Java
  source code and Markdown-based writing) in a Git repository using GitHub
  Classroom. Please verify that you have saved your source code in your Git
  repository by using \command{git status} to ensure that everything is updated.
  You can see if your assignment submission meets the established correctness
  requirements by using the provided checking tools on your local computer and
  in checking the commits in GitHub.

\item {\bf Keep all of your files}. Don't delete your programs, output files,
  and written reports after you submit them through GitHub; you will need them
  again when you study for the quizzes and examinations and work on the other
  laboratory, practical, and final project assignments.

\item {\bf Explore teamwork and technologies}. While certain aspects of the
  laboratory assignments will be challenging for you, each part is designed to
  give you the opportunity to learn both fundamental concepts in the field of
  computer science and explore advanced technologies that are commonly employed
  at a wide variety of companies. To explore and develop new ideas, you should
  regularly communicate with your team and/or the technical leaders.

\item {\bf Hone your technical writing skills}. Computer science assignments
  require to you write technical documentation and descriptions of your
  experiences when completing each task. Take extra care to ensure that your
  writing is interesting and both grammatically and technically correct,
  remembering that computer scientists must effectively communicate and
  collaborate with their team members and the student technical leaders and
  course instructor.

\item {\bf Review the Honor Code on the syllabus}. While you may discuss your
  assignments with others, copying source code or writing is a violation of
  Allegheny College's Honor Code.

\end{itemize}

\section*{Reading Assignment}

If you have not done so already, please read all of the relevant ``GitHub
Guides'', available at \url{https://guides.github.com/}, that explain how to use
many of the features that GitHub provides. In particular, please make sure that
you have read guides such as ``Mastering Markdown'' and ``Documenting Your
Projects on GitHub''; each of them will help you to understand how to use both
GitHub and GitHub Classroom. To do well on this assignment, you should also read
Chapter 2 in the textbook, paying particularly close attention to Section 2.5's
content about generics. Please see the instructor or one of the teaching
assistants if you have questions on this reading assignment.

\section*{Accessing the Laboratory Assignment on GitHub}

To access the laboratory assignment, you should go into the
\channel{\#announcements} channel in our Slack workspace and find the
announcement that provides a link for it. Copy this link and paste it into a web
browser. Now, you should accept the assignment and see that GitHub Classroom
created a new GitHub repository for you to access the assignment's starting
materials and to store the completed version of your assignment. Specifically,
to access your new GitHub repository for this assignment, please click the green
``Accept'' button and then click the link that is prefaced with the label ``Your
assignment has been created here''. If you accepted the assignment and correctly
followed these steps, you should have created a GitHub repository with a name
like
``Allegheny-Computer-Science-101-Fall-2019/computer-science-101-fall-2019-lab-3-gkapfham''.
Unless you provide the instructor with documentation of the extenuating
circumstances that you are facing, not accepting the assignment means that you
automatically receive a failing grade for it.

Before you move to the next step of this assignment, please make sure that you
read all of the content on the web site for your new GitHub repository, paying
close attention to the technical details about the commands that you will type
and the output that your program must produce. Now you are ready to download the
starting materials to your laboratory computer. Click the ``Clone or download''
button and, after ensuring that you have selected ``Clone with SSH'', please
copy this command to your clipboard. To enter into your course directory you
should now type \command{cd cs101F2019}. By typing \command{git clone} in your
terminal and then pasting in the string that you copied from the GitHub site you
will download all of the code for this assignment. For instance, if the course
instructor ran the \command{git clone} command in the terminal, it would look
like:

\begin{lstlisting}
  git clone git@github.com:Allegheny-Computer-Science-101-F2019/computer-science-101-fall-2019-lab-3-gkapfham.git
\end{lstlisting}

After this command finishes, you can use \command{cd} to change into the new
directory. If you want to \step{go back} one directory from your current
location, then you can type the command \command{cd ..}. Please continue to use
the \command{cd} and \command{ls} commands to explore the files that you
automatically downloaded from GitHub. What files and directories do you see?
What do you think is their purpose? Spend some time exploring, sharing your
discoveries with a neighbor and a \mbox{technical leader}.

\section*{Implementing and Testing a Generic Array Reversal Program}

Please use a text editor to study the source code of the \mainprogram{} class,
noticing that some of its methods are incomplete. Now, can you draw a picture
showing how these classes (e.g., both of the test suites and the main class) are
related? For instance, you should notice that the \mainprogram{} has a test
suite called \testprogram{} that is written in the JUnit testing framework;
please refer to Section 1.9 and the comments in the source code of this test
suite to learn more about how it works. Remember, a JUnit test suite has
individual test cases, written as Java methods, that each work to establish a
confidence in the correctness of \mainprogram{}. Once you understand how the
test suite assesses the \mainprogram's correctness, you should notice that this
class is missing an implementation of the \program{reverse} method. Please refer
to page 95 of the textbook for a suitable implementation of this method. What is
the meaning of the \program{<T>} and \program{T[]} notation in the signature of
this method? Why do you think that this reversal method is declared to be
\program{void}? Finally, you must enhance the \program{main} method in the
\mainprogram{} class so that it produces the output that is given in
Figure~\ref{fig:output}.

If you want to \step{build} your program you can type the command \gradlebuild{}
in your Docker container, thereby causing the Java compiler to check your
program for errors and get it ready to run. Please note that when you first try
to build the program you will see the following messages in your terminal
window. What is the meaning of these messages? What does they tell you about the
correctness of the program? Where do they suggest that you should start working
on this assignment? In this output what is the meaning of the notation
\command{TestReverser.java:26}?

\vspace*{-.075in}
\begin{verbatim}
  labthree.TestReverser > testSentenceReversalWithOneSentence FAILED
      java.lang.AssertionError at TestReverser.java:26
  labthree.TestReverser > testSentenceReversalWithOneInteger FAILED
      java.lang.AssertionError at TestReverser.java:33
  labthree.TestReverser > testSentenceReversalWithTwoIntegers FAILED
      java.lang.NullPointerException at TestReverser.java:41
  labthree.TestReverser > testSentenceReversalWithTwoSentences FAILED
      java.lang.NullPointerException at TestReverser.java:50
  labthree.TestReverser > testSentenceReversalWithManySentences FAILED
      java.lang.NullPointerException at TestReverser.java:62
  labthree.TestReverser > testIntegerReversalWithManyIntegers FAILED
      java.lang.NullPointerException at TestReverser.java:74
  labthree.TestSentence > testFirstSentenceHasCorrectSentence PASSED
  labthree.TestSentence > testTwoSentencesHaveCorrectIdentifiers PASSED
  labthree.TestSentence > testFirstSentenceHasFirstIdentifier PASSED
\end{verbatim}
\vspace*{-.075in}

Please note that once you implement the required methods correctly, the program
will not produce these error messages during testing. As you try to fix these
error messages, go back to your text editor and figure out how the program is
incorrect and then fix it. Once you have solved the problem, make a note of the
error and the solution for resolving it. Remember, you can run the tests for the
classes in \mainprogram{} by typing \gradletest{} in the Docker container.
Again, please make sure that you understand the purpose of each of these test
cases and that you consider adding more test cases. You should incrementally
complete this assignment, making small changes and regularly saving and
committing the Java source code and Markdown-based technical writing.

% If you cannot build \mainprogram{} correctly, then please talk with the course
% instructor or a student technical leader.

Once you resolve all of the building and testing errors, you can run your
program by typing \gradlerun{} in the Docker container---this is the ``execute''
step that will run your program and produce the designated output. You can see
if your program is producing the desired output by looking at
Figure~\ref{fig:output}. Do you see that it produces \mainprogramoutput{} lines
of output? Can you see that it performs the array reversal correctly? If not,
then repair the program and re-build and re-run it. Once the program runs,
please reflect on this process. What step did you find to be the most
challenging? Why? You should write your reflections in a file, called
\reflection{}, that uses the Markdown writing language. To complete this aspect
of the assignment, you should write multiple high-quality paragraphs that report
on your experiences. Your \reflection{} file should also contain answers to
other questions about \mainprogram{}'s implementation and testing.

\begin{figure}[t]
  \centering
  \begin{verbatim}
  Printing Sentences before Reversal ...
  (0, Sentence 0)
  (1, Sentence 1)
  (2, Sentence 2)
  (3, Sentence 3)
  (4, Sentence 4)
  ... Done Printing Sentences before Reversal.
  Printing Sentences after Reversal ...
  (4, Sentence 4)
  (3, Sentence 3)
  (2, Sentence 2)
  (1, Sentence 1)
  (0, Sentence 0)
  ... Done Printing Sentences after Reversal.
  Printing Integers before Reversal ...
  0
  1
  2
  3
  4
  ... Done Printing Integers before Reversal.
  Printing Integers after Reversal ...
  4
  3
  2
  1
  0
  ... Done Printing Integers after Reversal.
  \end{verbatim}
  \vspace*{-.25in}
  \caption{The Expected Output of the General-Purpose \mainprogram{}.}~\label{fig:output}
  \vspace*{-.25in}
\end{figure}

\section*{Checking the Correctness of Your Program and Writing}

As verified by the Checkstyle tool, the source code for the \mainprogram{} and
the other Java files must adhere to all of the requirements in the Google Java
Style Guide available at
\url{https://google.github.io/styleguide/javaguide.html}. The Markdown file that
contains your reflection must adhere to the standards described in the Markdown
Syntax Guide at \url{https://guides.github.com/features/mastering-markdown/}.
Finally, your \reflection{} file should adhere to the Markdown standards
established by the \step{Markdown linting} tool available at
\url{https://github.com/markdownlint/markdownlint/}. Instead of requiring you to
manually check that your deliverables adhere to these industry-accepted
standards, GatorGrader makes it easy for you to automatically check if your
submission meets the standards for correctness. For instance, automated tools
will run the provided JUnit tests, check to ensure that \mainprogram{} produces
\mainprogramoutput{} lines of output, and see that you commit to GitHub a
sufficient number of times when completing this assignment.

To get started with the use of GatorGrader, type the command \gatorgraderstart{}
in your Docker container. If your laboratory work does not meet all of the
assignment's requirements, then you will see a summary of the failing checks
along with a statement giving the percentage of checks that are currently
passing. If you do have mistakes in your assignment, then you will need to
review GatorGrader's output, find the mistake, and try to fix it, consulting
your text book and course notes as needed. Once your program is building
correctly, fulfilling at least some of the assignment's requirements, you should
transfer your files to GitHub using the \gitcommit{} and \gitpush{} commands.
For example, if you want to signal that the \mainprogramsource{} file has been
changed and is ready for transfer to GitHub you would first type
\gitcommitmainprogram{} in your terminal, followed by typing \gitpush{} and
checking to see that the transfer to GitHub is successful.

After the course instructor enables \step{continuous integration} with a system
called Travis CI, when you use the \gitpush{} command to transfer your source
code to your GitHub repository, Travis CI will initialize a \step{build} of your
assignment, checking to see if it meets all of the requirements. If both your
source code and writing meet all of the established requirements, then you will
see a green \checkmark{} in the listing of commits in GitHub after awhile. If
your submission does not meet the requirements, a red \naughtmark{} will appear
instead. The instructor will reduce a student's grade for this assignment if the
red \naughtmark{} appears on the last commit in GitHub immediately before the
assignment's due date. Yet, if the green \checkmark{} appears on the last commit
in your GitHub repository, then you satisfied all of the main checks, thereby
allowing the course instructor to evaluate other aspects of your source code and
writing, as further described in the \step{Evaluation} section of this
assignment sheet. Unless you provide the instructor with documentation of the
extenuating circumstances that you are facing, no late work will be considered
towards your grade for this laboratory assignment.

Remember that if you are completing this laboratory assignment on your own
laptop the GatorGrader program will only run correctly if you run it in a Docker
container.
%
Note that assignment checking will still work with Travis CI even if you do not
have Docker Desktop installed on your laptop.
%
Please talk with a technical leader or the instructor if you cannot use Docker
Desktop.

\section*{Summary of the Required Deliverables}

\noindent Students do not need to submit printed source code or technical
writing for any assignment in this course. Instead, this assignment invites you
to submit, using GitHub, the following deliverables.

\vspace*{-.075in}
\begin{enumerate}

  \setlength{\itemsep}{0in}

\item Stored in a Markdown file called \reflection{}, a multiple-paragraph
  response containing the requested Java source code explanations and the fenced
  code blocks.

\item A properly documented, well-formatted, and correct version of
  \mainprogramsource{} that meets all of the established requirements and
  produces the desired output.

\item A properly documented, well-formatted, and correct version of any
  additional Java files you changed to complete this assignment (e.g.,
  \testprogramsource{}).

\end{enumerate}
\vspace*{-.1in}

\section*{Evaluation of Your Laboratory Assignment}

Using a report that the instructor shares with you through the commit log in
GitHub, you will privately received a grade on this assignment and feedback on
your submitted deliverables. Your grade for the assignment will be a function of
the whether or not it was submitted in a timely fashion and if your program
received a green \checkmark{} indicating that it met all of the requirements.
Other factors will also influence your final grade on the assignment. In
addition to studying the efficiency and effectiveness and documentation of your
Java source code, the instructor will also evaluate the correctness of your
technical writing. If your submission receives a red \naughtmark{}, the
instructor will reduce your grade for the assignment. Finally, please remember
to read your GitHub repository's \program{README.md} file for a description of
the four grades that you will receive for this laboratory assignment.

\section*{Adhering to the Honor Code}

In adherence to the Honor Code, students should complete this assignment on an
individual basis. While it is appropriate for students in this class to have
high-level conversations about the assignment, it is necessary to distinguish
carefully between the student who discusses the principles underlying a problem
with others and the student who produces assignments that are identical to, or
merely variations on, someone else's work. Deliverables (e.g., Java source code
or Markdown-based technical writing) that are nearly identical to the work of
others will be taken as evidence of violating the \mbox{Honor Code}. Please see
the course instructor if you have questions about this policy.

\end{document}
