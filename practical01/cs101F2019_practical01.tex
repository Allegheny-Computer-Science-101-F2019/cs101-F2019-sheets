\documentclass[11pt]{article}

% NOTE: The "Edit" sections are changed for each assignment

% Edit these commands for each assignment

\newcommand{\assignmentduedate}{September 2}
\newcommand{\assignmentassignedate}{August 30}
\newcommand{\assignmentnumber}{One}

\newcommand{\labyear}{2019}
\newcommand{\labdueday}{Monday}
\newcommand{\labassignday}{Friday}
\newcommand{\labtime}{9:00 am}

\newcommand{\assigneddate}{Assigned: \labassignday, \assignmentassignedate, \labyear{} at \labtime{}}
\newcommand{\duedate}{Due: \labdueday, \assignmentduedate, \labyear{} at \labtime{}}

% Edit these commands to give the name to the main program

\newcommand{\mainprogram}{\lstinline{Swap}}
\newcommand{\mainprogramsource}{\lstinline{src/main/java/practicalone/Swap.java}}

% Edit these commands to give the name to the test suite

\newcommand{\testprogram}{\lstinline{TestSwap}}
\newcommand{\testprogramsource}{\lstinline{src/test/java/practicalone/TestSwap.java}}

% Commands to describe key development tasks

% --> Running gatorgrader.sh
\newcommand{\gatorgraderstart}{\command{gradle grade}}
\newcommand{\gatorgradercheck}{\command{gradle grade}}

% --> Compiling and running and testing program with gradle
\newcommand{\gradlebuild}{\command{gradle build}}
\newcommand{\gradletest}{\command{gradle test}}
\newcommand{\gradlerun}{\command{gradle run}}

% Commands to describe key git tasks

% NOTE: Could be improved, problems due to nesting

\newcommand{\gitcommitfile}[1]{\command{git commit #1}}
\newcommand{\gitaddfile}[1]{\command{git add #1}}

\newcommand{\gitadd}{\command{git add}}
\newcommand{\gitcommit}{\command{git commit}}
\newcommand{\gitpush}{\command{git push}}
\newcommand{\gitpull}{\command{git pull}}

\newcommand{\gitcommitmainprogram}{\command{git commit src/main/java/practicalone/Swap.java -m "Your
descriptive commit message"}}

% Use this when displaying a new command

\newcommand{\command}[1]{``\lstinline{#1}''}
\newcommand{\program}[1]{\lstinline{#1}}
\newcommand{\url}[1]{\lstinline{#1}}
\newcommand{\channel}[1]{\lstinline{#1}}
\newcommand{\option}[1]{``{#1}''}
\newcommand{\step}[1]{``{#1}''}

\usepackage{pifont}
\newcommand{\checkmark}{\ding{51}}
\newcommand{\naughtmark}{\ding{55}}

\usepackage{listings}
\lstset{
  basicstyle=\small\ttfamily,
  columns=flexible,
  breaklines=true
}

\usepackage{fancyhdr}

\usepackage[margin=1in]{geometry}
\usepackage{fancyhdr}

\pagestyle{fancy}

\fancyhf{}
\rhead{Computer Science 101}
\lhead{Practical Assignment \assignmentnumber{}}
\rfoot{Page \thepage}
\lfoot{\duedate}

\usepackage{titlesec}
\titlespacing\section{0pt}{6pt plus 4pt minus 2pt}{4pt plus 2pt minus 2pt}

\newcommand{\labtitle}[1]
{
  \begin{center}
    \begin{center}
      \bf
      CMPSC 101\\Data Abstraction\\
      Fall 2019\\
      \medskip
    \end{center}
    \bf
    #1
  \end{center}
}

\begin{document}

\thispagestyle{empty}

\labtitle{Practical \assignmentnumber{} \\ \assigneddate{} \\ \duedate{}}

% Slack for this course:

% https://join.slack.com/t/cmpsc101fall2019/signup

\section*{Objectives}

To learn how to install and use a contained-based platform, called Docker, to
support the completion of technical activities (e.g., building and running a
Java program) during the class, laboratory, and practical sessions. In addition
to learning how to install and use a text editor such as Atom or VS Code,
student will also learn how to install a package manager such as Brew or
Chocolatey. After you install a package manager, you will use it to further
install programs like a terminal window and the Git command-line client. As
necessary, students may optionally configure their laptops and GitHub accounts
--- although a later assignment will also review the steps needed to complete
this task. Throughout this assignment, students will also learn how to use Slack
to support communication with each other, the student technical leaders, and the
course instructor.

\section*{Suggestions for Success}

\begin{itemize}
  \setlength{\itemsep}{0pt}

\item {\bf Follow each step carefully}. Slowly read each sentence in the
  assignment sheet, making sure that you precisely follow each instruction. Take
  notes about each step that you attempt, recording your questions and ideas and
  the challenges that you faced. If you are stuck, then please tell a teaching
  assistant or instructor what assignment step you recently completed.

\item {\bf Regularly ask and answer questions}. Please log into Slack at the
  start of a laboratory or practical session and then join the appropriate
  channel. If you have a question about one of the steps in an assignment, then
  you can post it to the designated channel. Or, you can ask a student sitting
  next to you or talk with a teaching assistant or the course instructor.

% \item {\bf Store your files in GitHub}. Starting with this practical assignment, you will be responsible for storing
%   all of your files (e.g., Java source code and Markdown-based writing) in a Git repository using GitHub Classroom.
%   Please verify that you have saved your source code in your Git repository by using \command{git status} to ensure that
%   everything is updated. You can see if your assignment submission meets the established correctness requirements by
%   using the provided checking tools on your local computer and in checking the commits in GitHub.

% \item {\bf Keep all of your files}. Don't delete your programs, output files,
%   and written reports after you submit them through GitHub; you will need them
%   again when you study for the quizzes and examinations and work on the other
%   laboratory, practical, and final project assignments.

% \item {\bf Back up your files regularly}. All of your files are regularly
%   backed-up to the servers in the Department of Computer Science and, if you
%   commit your files regularly, stored on GitHub. However, you may want to use a
%   flash drive, Google Drive, or your favorite backup method to keep an extra
%   copy of your files on reserve. In the event of any type of system failure, you
%   are responsible for ensuring that you have access to a recent backup copy of
%   all your files.

\item {\bf Explore teamwork and technologies}. While certain aspects of the
  course assignments will be challenging for you, each part is designed to give
  you the opportunity to learn both fundamental concepts in the field of
  computer science and explore advanced technologies that are commonly employed
  at a wide variety of companies. To explore and develop new ideas, you should
  regularly communicate with your team and/or the teaching assistants and
  tutors.

\item {\bf Hone your technical writing skills}. Computer science assignments
  require to you write technical documentation and descriptions of your
  experiences when completing each task. Take extra care to ensure that your
  writing is interesting and both grammatically and technically correct,
  remembering that computer scientists must effectively communicate and
  collaborate with their team members and the tutors, teaching assistants, and
  course instructor.

\item {\bf Review the Honor Code on the syllabus}. While you may discuss your
  assignments with others, copying source code or writing is a violation of
  Allegheny College's Honor Code.

\end{itemize}

\section*{Reading Assignment}

If you have not done so already, please read all of the relevant ``GitHub
Guides'', available at \url{https://guides.github.com/}, that explain how to use
many of the features that GitHub provides. In particular, please make sure that
you have read guides such as ``Mastering Markdown'' and ``Documenting Your
Projects on GitHub''; each of them will help you to understand how to use both
GitHub and GitHub Classroom. In addition to reading all of the relevant
documentation about Docker that is available at \url{https://docs.docker.com/},
students should review the documentation about their chosen text editor (e.g.,
Vim, Atom, or VS Code) and package manager (e.g., Chocolatey or Brew); to
complete this part of the reading assignment, please visit the web site of each
tool.

\section*{Summary of the Required Deliverables}

\noindent Students do not need to submit printed source code or technical
writing for any assignment in this course. Instead, this assignment invites you
to submit, using GitHub, the following deliverables.

\vspace*{-.025in}

\begin{enumerate}

  \setlength{\itemsep}{0in}

\item A properly documented, well-formatted, and correct version of
  \mainprogramsource{} that meets the established requirements and produces the
  desired output.

\item A properly documented, well-formatted, and correct version of
  \testprogramsource{} that meets the established requirements and produces the
  desired output.

\end{enumerate}

\vspace*{-.075in}

\section*{Evaluation of Your Practical Assignment}

Using a report that the instructor shares with you through the commit log in
GitHub, you will privately received a grade on this assignment and feedback on
your submitted deliverables. Your grade for the assignment will be a function
of the whether or not it was submitted in a timely fashion and if your program
received a green \checkmark{} indicating that it met all of the requirements.
Other factors will also influence your final grade on the assignment. In
addition to studying the efficiency and effectiveness of your Java source code,
the instructor will also evaluate the accuracy of both your technical writing
and the comments in your source code. If your submission receives a red
\naughtmark{}, the instructor will reduce your grade for the assignment while
still considering the regularity with which you committed to your GitHub
repository and the overall quality of your partially completed work. Please see
the instructor if you have questions about the evaluation of this practical
assignment.

\end{document}
