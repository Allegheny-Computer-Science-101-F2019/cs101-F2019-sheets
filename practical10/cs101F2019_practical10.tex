\documentclass[11pt]{article}

% NOTE: The "Edit" sections are changed for each assignment

% Edit these commands for each assignment

\newcommand{\assignmentduedate}{December 9}
\newcommand{\assignmentassignedate}{December 6}
\newcommand{\assignmentnumber}{Nine}

\newcommand{\labyear}{2019}
\newcommand{\labday}{Friday}
\newcommand{\labdueday}{Monday}
\newcommand{\labtime}{9:00 am}

\newcommand{\assigneddate}{Assigned: \labday, \assignmentassignedate, \labyear{} at \labtime{}}
\newcommand{\duedate}{Due: \labdueday, \assignmentduedate, \labyear{} at \labtime{}}

% Edit these commands to give the name to the main program

\newcommand{\mainprogram}{\lstinline{WordCount}}
\newcommand{\mainprogramsource}{\lstinline{src/main/java/practicalnine/count/WordCount.java}}

% Edit these commands to give the main program's output details

\newcommand{\mainprogramoutput}{four}

% Edit this commands to describe key deliverables

\newcommand{\reflection}{\lstinline{writing/reflection.md}}

% Commands to describe key development tasks

% --> Running gatorgrader.sh
\newcommand{\gatorgraderstart}{\command{gradle grade}}
\newcommand{\gatorgradercheck}{\command{gradle grade}}

% --> Compiling and running and testing program with gradle
\newcommand{\gradlebuild}{\command{gradle build}}
\newcommand{\gradletest}{\command{gradle test}}
\newcommand{\gradlerun}{\command{gradle run}}

% Commands to describe key git tasks

% NOTE: Could be improved, problems due to nesting

\newcommand{\gitcommitfile}[1]{\command{git commit #1}}
\newcommand{\gitaddfile}[1]{\command{git add #1}}

\newcommand{\gitadd}{\command{git add}}
\newcommand{\gitcommit}{\command{git commit}}
\newcommand{\gitpush}{\command{git push}}
\newcommand{\gitpull}{\command{git pull}}

\newcommand{\gitcommitmainprogram}{\command{git commit src/main/java/practicalseven/list/DoublyLinkedList.java -m "Your
descriptive commit message"}}

% Use this when displaying a new command

\newcommand{\command}[1]{``\lstinline{#1}''}
\newcommand{\program}[1]{\lstinline{#1}}
\newcommand{\url}[1]{\lstinline{#1}}
\newcommand{\channel}[1]{\lstinline{#1}}
\newcommand{\option}[1]{``{#1}''}
\newcommand{\step}[1]{``{#1}''}

\usepackage{pifont}
\newcommand{\checkmark}{\ding{51}}
\newcommand{\naughtmark}{\ding{55}}

\usepackage{listings}
\lstset{
  basicstyle=\small\ttfamily,
  columns=flexible,
  breaklines=true
}

\usepackage{fancyhdr}

\usepackage[margin=1in]{geometry}
\usepackage{fancyhdr}

\pagestyle{fancy}

\fancyhf{}
\rhead{Computer Science 101}
\lhead{Practical Assignment \assignmentnumber{}}
\rfoot{Page \thepage}
\lfoot{\duedate}

\usepackage{titlesec}
\titlespacing\section{0pt}{6pt plus 4pt minus 2pt}{4pt plus 2pt minus 2pt}

\newcommand{\labtitle}[1]
{
  \begin{center}
    \begin{center}
      \bf
      CMPSC 101\\Data Abstraction\\
      Fall 2019\\
      \medskip
    \end{center}
    \bf
    #1
  \end{center}
}

\begin{document}

\thispagestyle{empty}

\labtitle{Practical \assignmentnumber{} \\ \assigneddate{} \\ \duedate{}}

\section*{Objectives}

This assignment invites you to study the source code of a program that is
inspired by one outlined in Chapter 10 of the textbook. Specifically, you will
study the source code and output of the \mainprogram{} program that combines the
use of the \program{Tree} and \program{HashMap} data structures. You will add
comments to the source code that clearly explain how the provided program works.
Interested students are also invited to add advanced features to this program
that, for instance, display the word with the maximum appearance count or
support the analysis of multiple input files.

\section*{Suggestions for Success}

\begin{itemize}
  \setlength{\itemsep}{0pt}

\item {\bf Follow each step carefully}. Slowly read each sentence in the
  assignment sheet, making sure that you precisely follow each instruction. Take
  notes about each step that you attempt, recording your questions and ideas and
  the challenges that you faced. If you are stuck, then please tell a technical
  leader or instructor what assignment step you recently completed.

\item {\bf Regularly ask and answer questions}. Please log into Slack at the
  start of a laboratory or practical session and then join the appropriate
  channel. If you have a question about one of the steps in an assignment, then
  you can post it to the designated channel. Or, you can ask a student sitting
  next to you or talk with a technical leader or the course instructor.

\item {\bf Store your files in GitHub}. Starting with this laboratory
  assignment, you will be responsible for storing all of your files (e.g., Java
  source code and Markdown-based writing) in a Git repository using GitHub
  Classroom. Please verify that you have saved your source code in your Git
  repository by using \command{git status} to ensure that everything is updated.
  You can see if your assignment submission meets the established correctness
  requirements by using the provided checking tools on your local computer and
  in checking the commits in GitHub.

\item {\bf Keep all of your files}. Don't delete your programs, output files,
  and written reports after you submit them through GitHub; you will need them
  again when you study for the quizzes and examinations and work on the other
  laboratory, practical, and final project assignments.

\item {\bf Explore teamwork and technologies}. While certain aspects of the
  laboratory assignments will be challenging for you, each part is designed to
  give you the opportunity to learn both fundamental concepts in the field of
  computer science and explore advanced technologies that are commonly employed
  at a wide variety of companies. To explore and develop new ideas, you should
  regularly communicate with your team and/or the student technical leaders.

\item {\bf Hone your technical writing skills}. Computer science assignments
  require to you write source code documentation and detailed notes of your
  experiences when completing each task. Take extra care to ensure that your
  writing is interesting and both grammatically and technically correct,
  remembering that computer scientists must effectively communicate and
  collaborate with their team members and the technical leaders and course
  instructor.

\item {\bf Review the Honor Code on the syllabus}. While you may discuss your
  assignments with others, copying source code or writing is a violation of
  Allegheny College's Honor Code.

\end{itemize}

\vspace*{-.2in}

\section*{Reading Assignment}

If you have not done so already, please read all of the relevant ``GitHub
Guides'', available at \url{https://guides.github.com/}, that explain how to use
many of the features that GitHub provides. In particular, please make sure that
you have read guides such as ``Mastering Markdown'' and ``Documenting Your
Projects on GitHub''; each of them will help you to understand how to use both
GitHub and GitHub Classroom. To do well on this assignment, you should also read
Chapters 8 and 10 in the textbook, paying particularly close attention to the
material about the \program{Tree} and \program{HashMap} data structures. Please
see the instructor with questions about these reading assignments.

\section*{Accessing the Practical Assignment on GitHub}

To access the laboratory assignment, you should go into the
\channel{\#announcements} channel in our Slack team and find the announcement
that provides a link for it. Copy this link and paste it into a web browser.
Now, you should accept the assignment and see that GitHub Classroom created a
new GitHub repository for you to access the assignment's starting materials and
to store the completed version of your assignment. Specifically, to access your
new GitHub repository for this assignment, please click the green ``Accept''
button and then click the link that is prefaced with the label ``Your assignment
has been created here''. If you accepted the assignment and correctly followed
these steps, you should have created a GitHub repository with a name like
``Allegheny-Computer-Science-101-Fall-2019/computer-science-101-fall-2019-practical-9-gkapfham''.

% Unless you provide the instructor with documentation of the extenuating
% circumstances that you are facing, not accepting the assignment means that you
% automatically receive a failing grade for it.

Before you move to the next step of this assignment, please make sure that you
read all of the content on the web site for your new GitHub repository, paying
close attention to the technical details about the commands that you will type
and the output that your program must produce. Now you are ready to download the
starting materials to your laboratory computer. Click the ``Clone or download''
button and, after ensuring that you have selected ``Clone with SSH'', please
copy this command to your clipboard. To enter into your course directory you
should now type \command{cd cs101F2019}. By typing \command{git clone} in your
terminal and then pasting in the string that you copied from the GitHub site you
will download all of the code for this assignment. For instance, if the course
instructor ran the \command{git clone} command in the terminal, it would look
like:

\begin{lstlisting}
  git clone git@github.com:Allegheny-Computer-Science-101-F2019/computer-science-101-fall-2019-practical-9-gkapfham.git
\end{lstlisting}

After this command finishes, you can use \command{cd} to change into the new
directory. If you want to \step{go back} one directory from your current
location, then you can type the command \command{cd ..}. Please continue to use
the \command{cd} and \command{ls} commands to explore the files that you
automatically downloaded from GitHub. What files and directories do you see?
What do you think is their purpose? Spend some time exploring, sharing your
discoveries with a neighbor and a \mbox{teaching assistant}.
Specifically, each student should ensure that they can understand how the
methods in this class are related.

\section*{Combining the Tree and HashMap Data Structures}

This practical assignment invites you to study and understand the
implementation of \mainprogram{} that can produces the output given in the
GitHub repository. Specifically, the \mainprogramsource{} provides an
implementation of the program that we effectively combines \program{Tree} and
\program{HashMap} data structures. For this assignment, you are invited to
study, understand, and comment this program's source code. Please make sure
that you can write comments that explain how this program uses the combination
of a \program{Tree} and a \program{HashMap} to output word counts for the
paragraph in a provided file. You should also add the required source code so
that the program produces the expected output. Don't forget to run the program
and paste its output at the bottom of the file! You should also observe that
the code uses the \program{java.util.TreeMap} --- how would the program's
output change if you replace that with a \program{java.util.HashMap} instead?
You should also be able to explain how the program performs input parsing and
sorting of the populated \program{TreeMap}. Finally, make sure that you can
explain this program output, noting how it illustrates the use of a
\program{HashMap}. Please see the course instructor with your questions about
this program!

\begin{verbatim}
(key, value) pairs sorted by value and key:

[already=1, append=1, associates=1, be=1, combining=1, create=1, either=1,
example=1, function=1, it=1, mapped=1, mapping=1, may=1, method=1, msg=1,
multiple=1, non=1, not=1, otherwise=1, remapping=1, removes=1, replaces=1,
result=1, results=1, specified=1, string=1, this=1, use=1, values=1, when=1,
for=2, given=2, if=2, key=2, of=2, to=2, associated=3, is=3, null=3, or=3, a=4,
value=4, with=4, the=6]
\end{verbatim}

\section*{Checking the Correctness of Your Program}

As verified by the Checkstyle tool, the source code for the \mainprogram{} and
all of the other Java files must adhere to all of the requirements in the Google
Java Style Guide available at
\url{https://google.github.io/styleguide/javaguide.html}. Instead of requiring
you to manually check that your deliverables adhere to these industry-accepted
standards, GatorGrader and Gradle make it easy for you to automatically check if
your submission meets the correctness requirements.

After the course instructor enables \step{continuous integration} with a system
called Travis CI, when you use the \gitpush{} command to transfer your source
code to your GitHub repository, Travis CI will initialize a \step{build} of your
assignment, checking to see if it meets all of the requirements. If both your
source code and documentation meet all of the established requirements, then you
will see a green \checkmark{} in the listing of commits in GitHub after awhile.
If your submission does not meet the requirements, a red \naughtmark{} will
appear instead. The instructor will reduce a student's grade for this assignment
if the red \naughtmark{} appears on the last commit in GitHub immediately before
the assignment's due date. Yet, if the green \checkmark{} appears on the last
commit in your GitHub repository, then you satisfied all of the main checks,
thereby allowing the course instructor to evaluate other aspects of your source
code and writing, as further described in the \step{Evaluation} section.

\section*{Summary of the Required Deliverables}

\noindent Students do not need to submit printed source code or technical
writing for any assignment in this course. Instead, this assignment invites you
to submit, using GitHub, the following deliverables.

\vspace*{-.1in}

\begin{enumerate}

  \setlength{\itemsep}{0in}

\item A properly documented and correct version of the \mainprogram{}. The
  comments in this source code should clearly explain how it uses both the
  \program{Tree} and \program{HashMap} to accomplish its task.

\end{enumerate}

\vspace*{-.2in}

\section*{Evaluation of Your Practical Assignment}

Using a report that the instructor shares with you through the commit log in
GitHub, you will privately received a grade on this assignment and feedback on
your submitted deliverables. Your grade for the assignment will be a function of
the whether or not it was submitted in a timely fashion and if your program
received a green \checkmark{} indicating that it met all of the requirements.
Other factors will also influence your final grade on the assignment. In
addition to studying the efficiency and effectiveness of your Java source code,
the instructor will also evaluate the accuracy of the technical writing in your
source code's comments. If your submission receives a red \naughtmark{}, the
instructor will reduce your grade for the assignment while still considering the
regularity with which you committed to your GitHub repository and the overall
quality of your partially completed work. Please see the instructor if you have
questions about the evaluation of this practical assignment.

% \section*{Adhering to the Honor Code}

% In adherence to the Honor Code, students should complete this assignment on an
% individual basis. While it is appropriate for students in this class to have
% high-level conversations about the assignment, it is necessary to distinguish
% carefully between the student who discusses the principles underlying a problem
% with others and the student who produces assignments that are identical to, or
% merely variations on, someone else's work. Deliverables (e.g., Java source code
% or Markdown-based technical writing) that are nearly identical to the work of
% others will be taken as evidence of violating the \mbox{Honor Code}. Please see
% the course instructor if you have questions about this policy.

\end{document}
